\documentclass[a4paper, twocolumn]{article}
\usepackage[magyar]{babel}
\usepackage[latin2]{inputenc}
\usepackage{t1enc}   %elválasztás!!!!

\usepackage[dvips]{graphicx}
\usepackage[dvips]{color}

% \textwidth 105mm
% \oddsidemargin -5mm
% \evensidemargin -5mm
% \textheight 155mm
% \topmargin -5mm
% \paperheight 297mm
% \paperwidth 210mm


\begin{document}
% author names, affiliation and email




\title{Wavelet}
\author{LaTeX Gyakorl\'o\\
Sz\'echenyi Istv\'an University, 9026 Gy\H{o}r, Egyetem t\'er 1.\\
xyz@abx.hu}
\maketitle
% make the title area


\begin{abstract}
aa
\end{abstract}



\section{Introduction\label{sec1}}
Wavelet trans. is a widely used mathematical method,  \cite{daub92}. As a first step a well chosen function $\psi$, i.e., 
\begin{equation}\label{elsoegyenlet}
[W_\psi f](a,b)=\frac{1}{\sqrt{|a|}} \int_{-\infty}^\infty \overline{\psi \left( \frac{x-b}{a} \right)}f(x)dx.
\end{equation}
\[
a=a\qquad a\quad a\ a\;a\,aa\!a\sin \cdot \ldots \cdots a\omega^3 t^2x_\alpha r
\]
\begin{eqnarray}
% \nonumber to remove numbering (before each equation)
  \sum_{k=1}^N \pi_k&=& 22 \label{eq1}\\
  \oint_A B dA &=& \int \Phi dl \nonumber\\
  \bigcup_{p=0}^N A_p&=& \sqrt{\emptyset} \label{eeeee}
\end{eqnarray}

\noindent \ref{elsoegyenlet} \ref{sec1} $\sum_{k=1}^N \pi_k= 22$ Here $b$ corresponds to the shift i.e., the position in time or space, and $a$ to the compression, i.e. the frequency. At the normal wavelet transformation th where $a = 2^{-j}$ and $b = k2^{-j}$. Values $k$ and $j$ 
\[
\begin{array}{lccc}
  a & b & c & d \\
  d & a & d & d \\
  s & k & w & e
\end{array}
\]
\[
\left(
  \begin{array}{cccc}
    a & c & g & r \\
    a & c & g & r \\
    a & c & g & r \\
    a & c & g & r \\
  \end{array}
\right)
\]
\begin{table}
\caption{asdfgh}
\label{asasas}
\begin{tabular}{|l|r|c|c|}
  \hline
  % after \\: \hline or \cline{col1-col2} \cline{col3-col4} ...
  a & y & c & 1 \\
  \hline
  a & a & 6 & 3 \\
  \cline{2-3}
  a & x & 7 & 2 \\
  \hline
\end{tabular}
\end{table}

\begin{figure}[h!]
\centerline{\includegraphics[width=8cm]{aaaa2.eps}}
\caption{Wavelet-transzformáció képre}
\label{fig1}
\end{figure}



\section{EDoF}




Ennek gyakorlati megval\'os\'{\i}t\'as\'ara Forster, Van de Ville, Berent, Sage \'es Unser \cite{a4} dolgozta ki az
\begin{eqnarray}
S^l_{n,m}(\Phi_j,\Phi_{j+1}) =& &  \nonumber \\
=\frac{Cov_{n,m}(W^{l-1}_A(\Phi_j),W^{l-1}_A(\Phi_{j+1})} {\sqrt{Var_{n,m}(W^{l-1}_A(\Phi_j))Var_{n,m}(W^{l-1}_A(\Phi_{j+1}))}}&&
\end{eqnarray}


%A kifejezésben $W^l_A(a)$ az $a$ kép $l$-edik szintű waveletranszformációja eredményeként keletkező alsó

\begin{equation}
e(x) = \frac{1}{N}||x||_1=\frac{1}{N}\sum_{i=1}^{N}|x_i|
\end{equation}

%\vfill
\vspace{-5 mm}

\begin{thebibliography}{1}
\bibitem{chui92} Chui, C. K.: An introduction to wavelets; Academic Press, San Diego, 1992. 266 p.
\bibitem{daub92} Daubechies, I.: Ten lectures on wavelets; Society for Industrial and Applied Mathematics, Pennsylvania, 1992. 365 p.
\bibitem{a3} Mallat, S.: A theory for multiresolution signal decomposition: the wavelet  representation; In: IEEE Transactions on Pattern Analysis and Machine Intelligence, Vol. 11, Issue 7, 1989. pp. 674-693.
\bibitem{a4} Gross, S.: Multi-observer survey on quality enhancement techniques
for still image acquisition in colonoscopy; 14th International Student Conference on
Electrical Engineering, Prague, 2010. BI2, 5 p.
\bibitem{a5} Forster, B.; Van de Ville, D.; Berent, J.; Sage, D.; Unser, M.: Complex wavelets for extended depth-of-field: A new method for the fusion of multichannel microscopy images; In: Microscopy Research and Technique, Vol. 65, 2004. pp. 33-42.
\bibitem{a6} Bradley, A. P.; Bamford, P. C.: A one-pass extended depth of field algorithm based on the over-complete discrete wavelet transform;  Image and Vision Computing '04 New Zealand, Akaroa, 2004. pp. 279-284.
\bibitem{a7} Kaftan, J.; Bell, A. A.; Seiler, C.; Aach, T.: Wavelet based denoising by correlation analysis for high dynamic range imaging; IEEE International Conference on Image Processing, Cairo, 2009. pp. 3857-3860.
\bibitem{a8} Chang, T.; Kuo, C.-C. J.: Texture analysis and classification with tree-structured wavelet transform; In: IEEE Transactions on Image Processing, Vol. 2, No. 4, 1993. pp. 429-441.

\end{thebibliography}

% that's all folks
\end{document}
